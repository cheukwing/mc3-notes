\documentclass[11pt]{article}
\usepackage{fullpage}
\usepackage{amsthm}
\usepackage{amsmath} \usepackage{amssymb}
\usepackage{graphicx}

\graphicspath{ {./imgs/} }

\setlength{\parindent}{0pt}

\title{Performance Engineering (CO339)}
\author{Michael Tsang}

\newtheorem{defn}{Definition}
\newtheorem{eg}{Example}
\newtheorem{theo}{Theorem}
\newtheorem{lem}{Lemma}

\begin{document}

\maketitle

\section{Motivation}
\textit{Performance engineering encompasses the techniques applied during a systems development life cycle to ensure the \textbf{non-functional requirements for performance} will be met.}

\subsection{System}
\begin{defn}
  A regularly interacting or independent group of items forming a unified whole.
\end{defn}

\begin{itemize}
  \item Made up from \textbf{components} that interact to achieve a \textbf{greater goal}.
  \item Applicable to many situations (generic).
  \item Goal is domain-agnostic.
  \item Components serve a data-management purpose, not a domain purpose.
\end{itemize}

\subsection{Success}
To define success, we need to define a \textbf{metric} and a \textbf{threshold}.

The metric could be throughput, latency, scalability, memory usage, energy consumption, TCO, elasticity, efficiency\dots.

We have two options when declaring success:
\begin{itemize}
  \item Setting an optimization \textbf{budget}.
  \item Setting an optimization \textbf{target}.
\end{itemize}

\subsection{Defining the Target}

\subsubsection{\textit{SMART} Requirements}
\begin{itemize}
  \item \textbf{Specific} - state exactly what is acceptable in numeric terms.
  \item \textbf{Measureable} - make sure what is stated can be measured.
  \item \textbf{Acceptable} - rigorous enough to guarantee success in reality.
  \item \textbf{Realizable} - lenient enough to allow implementation.
  \item \textbf{Thorough} - all necessary aspects of the system are specified.
\end{itemize}

\subsubsection{Quality-of-Service (QoS) Objectives}
\begin{defn}
  Statistical properties of a metric that shall hold for the system.
\end{defn}
\begin{itemize}
  \item Can include pre-conditions.
  \item Could conflict with functional requirements.
\end{itemize}

\begin{eg}
  The framerate of the game will, on average, be higher than 60 frames/s if run on a GPU with 50 GFlops or more.
\end{eg}

\subsubsection{Service-Level Agreements (SLA)}
\begin{defn}
  Formal, legal contracts specifying QoS objectives, as well as penalities for violations.
\end{defn}
\begin{itemize}
  \item Needs enforcement through monitoring.
\end{itemize}

\begin{eg}
  Trading orders shall not exceed 1ms response time. 
  In case of violation, the user is eligible for a 10\% credit towards fees.
\end{eg}


\subsubsection{Monitoring}
\begin{itemize}
  \item Constant monitoring is required to enforce SLAs.
    \begin{itemize}
      \item Observe system performance.
      \item Collect statistics.
      \item Analyze data.
      \item Report SLA violations.
    \end{itemize}
  \item Monitoring can incur costs $\rightarrow$ often not continuous.
\end{itemize}

\subsection{Fulfilling Performance Requirements}
Examples of performance evaluation techniques are:
\begin{itemize}
  \item Simulation.
  \item Measuring.
  \item Analytical modeling.
  \item Hybrids, e.g.\ measure then model; model and simulate\dots
\end{itemize}

\subsubsection{Measuring}
\begin{itemize}
  \item Performed on a prototype or on the final system.
  \item Closely related to monitoring, promises good accuracy.
  \item Often based on instrumentation.
  \item Costly, and can be difficult.
\end{itemize}

\subsubsection{Benchmark}
Two step process:
\begin{enumerate}
  \item Get the system into a predefined state.
  \item Perform a series of operations (the workload) while measuring performance.
\end{enumerate}

\subsubsection{Workloads}
\begin{itemize}
  \item \textbf{Batch} workloads:
    \begin{itemize}
      \item Program has access to entire batch at start.
      \item Useful when metric is throughput.
      \item Simple, generator performance does not matter.
    \end{itemize}
  \item \textbf{Interactive} workloads:
    \begin{itemize}
      \item Work generated piece by piece (randomly).
      \item Useful when metric is latency.
      \item Generator should be at least as fast as system under evaluation.
    \end{itemize}
  \item \textbf{Hybrids}:
    \begin{itemize}
      \item Sample random queries from a predefined work set.
    \end{itemize}
\end{itemize}

\subsubsection{Parameters}
\begin{itemize}
  \item \textbf{System Parameters} - generally do not change (e.g.\ caches; CPU instruction costs).
  \item \textbf{Workload Parameters} - may change, even while system is running (e.g.\ users; available memory).
\end{itemize}

\subsection{Interpreting Performance Metrics}
\begin{itemize}
  \item A single number is generally meaningless, there is too much noise in modern computer systems.
  \item We aggregate multiple runs and report some measure of variance - \textbf{statistics}.
  \item We can represent statistics with charts, e.g.\ box and whisker charts.
\end{itemize}

\begin{defn}
  \textbf{Utilization} is the percentage of a resource that is used to perform a service.
\end{defn}

\begin{defn}
  \textbf A {bottleneck} is the resource with the highest utilization.
\end{defn}

\subsection{Improving Performance}
We compare alternative designs (\textbf{development}) or select a close-to-optimal value for a parameter (\textbf{tuning}).

\subsubsection{Analytical Modeling}
\begin{defn}
  A set of equations describing the mathematical relationship between performance parameters and performance metrics.
\end{defn}

We model a dynamic system using \textbf{static equations}.
This is the key distinction to simulation, which is dynamic.

Analytical models are fast and allow what-if analysis, which simplifies tuning.
If we can build an analytical model, we have understood the system.

\subsubsection{Parameter Tuning}
Workload parameters are not usually under our control.
\begin{defn}
  \textbf{Parameter tuning} involves finding the vector in the parameter space that either \textbf{minimizes the resource consumption} or \textbf{maximizes a performance metric}. 
\end{defn}
We need to explore the parameter space, which is expensive.
Analytical models help to accelerate this process immensely.

\subsubsection{Tradeoffs}
Sometimes consumption of an expensive or non-scalable resource can be reduced by using more of a cheaper one - this may require changes to the system.

\end{document}
