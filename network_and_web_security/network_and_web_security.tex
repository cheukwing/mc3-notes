\documentclass[11pt]{article}
\usepackage{fullpage}
\usepackage{amsthm}
\usepackage{amsmath} \usepackage{amssymb}
\usepackage{graphicx}

\graphicspath{ {./imgs/} }

\setlength{\parindent}{0pt}

\title{Network and Web Security (CO331)}
\author{Michael Tsang}

\newtheorem{defn}{Definition}
\newtheorem{eg}{Example}
\newtheorem{theo}{Theorem}
\newtheorem{lem}{Lemma}

\begin{document}

\maketitle
\section{Cybersecurity}
Attacking a system is easy, only \textbf{one way} in is needed.

Defending a system is hard:
\begin{itemize}
  \item Defenses may interfere with business goals.
  \item Laws need to be enforced across borders.
  \item Targets are interconnected devices running vulnerable software.
  \item Hard to identify attackers.
  \item \textbf{All fronts} must be protected.
\end{itemize}

\subsection{Human Factors}
\begin{itemize}
  \item Social engineering.
  \item Weak passwords.
  \item Insider threats.
  \item Coercion.
\end{itemize}

\section{Vulnerabilities}
\begin{defn}
  \textbf{Vulnerabilities} are software bugs that attackers can exploit in order to compromise computers.
\end{defn}

\begin{defn}
  \textbf{Exploits} are pieces are software that take advantage of Vulnerabilities in order to access or infect a computer.
\end{defn}

\begin{defn}
  A \textbf{zero day} vulnerability or exploit is one that is unknown to the software vendor.
\end{defn}

Finders of zero-days could do one of four things:
\begin{enumerate}
  \item \textbf{Fix}.
  \item \textbf{Sell}.
  \item \textbf{Disclose}.
  \item \textbf{Exploit}.
\end{enumerate}

\subsection{Advisories}
\begin{itemize}
  \item Security advisories or bulletins publicly disclose new vulnerabilities.
  \item All published vulnerabilities are classified and given a unique ID.
  \item Once a vulnerability is public, proof of concept exploits may become available.
\end{itemize}

\subsection{Ethics}
If we discover a vulnerability, we can choose whether or not we disclose it:
\begin{itemize}
  \item \textbf{Full disclosure} - make the details public, open to public scrutiny:
    \begin{itemize}
      \item Preferred by security researchers and the open source community.
      \item May expose users to attack until patched - attackers may already know about it.
    \end{itemize}
  \item \textbf{Responsible disclosure} - affected vendor decides when to release information, and how much:
    \begin{itemize}
      \item Preferred by vendors.
      \item End-users will not develop their own patches.
      \item Can lead to longer time between discovery to fix.
    \end{itemize}
  \item \textbf{Non disclosure} - keep secret:
    \begin{itemize}
      \item Preferred by exploiters and vendors unwilling to invest in fixes.
      \item ``Security by obscurity''.
    \end{itemize}
\end{itemize}

\subsection{Fun and Profit}
\begin{itemize}
  \item Bug bounty programs:
    \begin{itemize}
      \item Some vendors offer rewards for finding vulnerabilities in their products.
      \item Companies may give explicit permission on what can be attacked, and what cannot.
    \end{itemize}
  \item Competitions (CTF).
  \item Vulnerability markets:
    \begin{itemize}
      \item Legitimate - run by security companies who buy vulnerabilities and re-sell to vendors.
      \item Black market - dark web.
    \end{itemize}
\end{itemize}

\section{Security Software Development Lifecycle (SSDLC)}
Security touchpoints in the software engineering are:
\begin{itemize}
  \item Abuse cases.
  \item Security requirements.
  \item Risk analysis:
    \begin{itemize}
      \item Threat modelling.
      \item Quantitative risk assessment.
    \end{itemize}
  \item Risk-based security analysis.
  \item Code review.
  \item Penetration testing.
  \item Security operations.
\end{itemize}
\subsection{Threat Modelling}
We use consistent visual syntax and focus on the system architecture.

\subsubsection{Data-Flow Diagram (DFD)}
\begin{figure}[htb!]
  \centering
  \caption{Data-flow diagram.}
  \includegraphics[scale=0.3]{dfd}
\end{figure}
\begin{itemize}
  \item Depict the flow of information across system components.
  \item External entities are out of our control.
  \item Trust boundaries help to establish what principal controls what.
  \item Attacks tend to cross trust boundaries.
\end{itemize}

\subsubsection{Identifying Threats: STRIDE}
For each element in the DFD, we ask what could go wrong:
\begin{itemize}
  \item \textbf{Spoofing} - pretending to be something or somebody else.
  \item \textbf{Tampering} - modifying without permission.
  \item \textbf{Repudiation} - denying to have done something.
  \item \textbf{Information Disclosure} - revealing information without permission.
  \item \textbf{Denial of Service} - preventing a system from providing a (timely) service.
  \item \textbf{Elevation of Privilege} - achieving more than what is intended.
\end{itemize}

Some threats may belong to more than one category.
We document threats by writing risk-based security tests.

\subsubsection{Identifying Threats: Attack Trees}
\begin{figure}[htb!]
  \centering
  \caption{Attack tree with the goal of authenticating with user credentials.}
  \includegraphics[scale=0.3]{attacktree}
\end{figure}
We create a tree structure with the following properties:
\begin{itemize}
  \item Root represents the attack goal, or asset being compromised.
  \item Children are steps to achieve goal.
  \item Leaves are concrete attacks.
  \item Siblings represent sufficient steps to achieve goal (step 1 or step 2).
  \item Special notation for sibilings which represent necessary steps (step 1 and step 2).
\end{itemize}

Attack trees are an alternative to STRIDE.
For each element in the DFD, if the goal of a tree is relevant, we start traversing the tree to identify possible attacks.
Attack trees capture domain-specific expertise and can be reused on different DFDs.

\subsubsection{Evaluating Threats}
We use the DREAD system to score each threat between 5 and 15, where higher scores imply greater risk, see figure \ref{fig:dread}.

\begin{figure}[htb!]
  \centering
  \caption{Scoring a system based on DREAD.}
  \label{fig:dread}
  \includegraphics[width=\textwidth]{dread}
\end{figure}

\subsubsection{Addressing each Threat}
To recommend a response, we use META:
\begin{itemize}
  \item \textbf{Mitigate} - make a threat hearder to exploit.
  \item \textbf{Eliminate} - remove the feature that was exposed to the threat.
  \item \textbf{Transfer} - let another party assume the risk.
  \item \textbf{Accept} - other options are impossible or impractical; keep track that the threat remains valid.
\end{itemize}

\section{Malicious Software (Malware)}
\begin{itemize}
  \item Format:
    \begin{itemize}
      \item Injected code added to a legitimate program.
      \item DLL called by legitimate program.
      \item Script run by an application.
      \item Standalone executable.
      \item Malicious code loaded in volatile memory.
    \end{itemize}
  \item Propagation:
    \begin{itemize}
      \item Installed by attacker - self replication, exploiting vulnerabilities.
      \item Installed by user - social engineering, compromised certificate.
    \end{itemize}
  \item Privileges:
    \begin{itemize}
      \item Root - owns the machine.
      \item User - limited damage, but can attempt elevation to root.
    \end{itemize}
\end{itemize}

\subsection{Malware Campaigns}
\begin{defn}
  \textbf{Malware campaigns} are targeted attacks aiming to infect the machine of a (or a few) particular victim(s).
\end{defn}

Advanced persistent threats (APTs) stealthily exploit a high-value target over time:
\begin{itemize}
  \item Wait for interesting information.
  \item Exfiltrate large database slowly.
  \item Gain access in order to exploit at a later date.
\end{itemize}

Generic attacks infect as many machines as possible:
\begin{itemize}
  \item Low-cost attacks with low chance of success.
  \item Botnet.
\end{itemize}

\subsection{Botnets}
\begin{itemize}
  \item One attacker (\textit{botmaster}) controls hundreds of thousands of infected machines (\textit{bots}).
  \item Bots connect to a \textit{command-and-control} (C\&C) server to receive instructions on what to do.
  \item Sophisticated topologies.
  \item Encrypted and stealthy communication of commands and results.
  \item Botmaster may keep changing IP.
\end{itemize}

\subsubsection{Goals}
\begin{itemize}
  \item \textbf{Data theft} - steal sensitive data.
  \item \textbf{Spam} - deliver unrequested email.
  \item \textbf{Distributed denial of service} (DDOS) - flood web servers with requests.
  \item \textbf{Credential stuffing} - attempt to login with leaked credentials to see which works.
  \item \textbf{Card cracking} - bruteforce missing information for card payments.
  \item \textbf{Network scanning} - attempt to probe other hosts.
  \item \textbf{Click fraud} - generate advertising revenue.
  \item \textbf{Cryptojacking} - mining cryptocurrencies.
\end{itemize}

\subsection{Spam}
Spammers are the marketers for affiliate programs that support online stores with the back-office functions.

\subsection{Exploit Kits}
Exploit kits are commoditised, commercial malware toolkits sold or rented out to criminals:
\begin{itemize}
  \item Automated vulnerability analysis, exploitation, and post-exploitation.
  \item Anti-virus evasion techniques.
  \item Operator subscribes to traffic from spam and malicious ads.
  \item Administration console to fine tune parameters and select victims.
\end{itemize}

\subsection{Malware Detection}
\begin{itemize}
  \item Anti-virus - detect malware just before or after infection:
    \begin{itemize}
      \item Impossible to have perfect anti-virus.
      \item Scan programs for \textit{signatures}, sequences of instructions typical of malware.
      \item Signatures can be obfuscated, so the new signatures need to be added to the anti-virus.
    \end{itemize}
  \item Blacklist web pages hosting phising and malware.
\end{itemize}

The attacker always has a window of opportunity before detection.

\subsection{Malware Analysis}
\begin{itemize}
  \item Malware samples are captured:
    \begin{itemize}
      \item Clean up after infection.
      \item \textit{Honeypot} - intentionally vulnerable machines.
    \end{itemize}
  \item Observe malware execution in a VM sandbox:
    \begin{itemize}
      \item Look at effects on storage, settings, network traffic, etc.
      \item Malware can kill logging processes in the guest OS.
      \item Malware can sometimes detect virtualization and behave differently.
    \end{itemize}
  \item Dynamic analysis:
    \begin{itemize}
      \item Extract a signature based on malware behaviour, not code - typically system call patterns.
      \item Malware could mix malicious with legitimate looking behaviour.
    \end{itemize}
\end{itemize}

\subsection{Malware Prevention}
\begin{itemize}
  \item Most common infection vectors are vulnerabilities and social engineering:
    \begin{itemize}
      \item Educate humans.
      \item Update and patch software in response to vulnerability disclosures.
      \item Firewalls and Intrusion Detection Systems.
    \end{itemize}
  \item Certified secure systems:
    \begin{itemize}
      \item Hardware and software should come with proof of correctness and/or security.
      \item Ongoing research.
    \end{itemize}
\end{itemize}

\section{Passwords}
Passwords are used in the protection of cryptographic keys and user authentication.

\subsection{Plain-text Passwords}
\begin{enumerate}
  \item Store credentials in password file.
  \item User presents username and password.
  \item Check if username is present.
  \item Check if password matches stored password.
  \item Grant or deny access.
\end{enumerate}

The password file is a valuable target for hackers.

\subsection{Encrypted Passwords}
\begin{defn}
  \textbf{Symmetric encryption}:
  \begin{align*}
    \text{Encrypt}(\text{key}, \text{ plaintext}) &= \text{ ciphertext} \\
    \text{Decrypt}(\text{key}, \text{ ciphertext}) &= \text{ plaintext} \\
  \end{align*}
\end{defn}

\begin{enumerate}
  \item Store encrypted credentials in a password file.
  \item User presents username and password.
  \item Check if username is present.
  \item Check that present password matches decryption of stored password.
  \item Grant or deny access.
\end{enumerate}

Key becomes a valuable target for hackers too.

\subsection{Hashed Passwords}
\begin{defn}
  \textbf{Cryptographic hashing}:
  \[
    \text{Hash}(\text{plaintext}) = \text{ hashvalue} 
  \]
  Theoretically a one-way function, cannot be reversed.
\end{defn}

\begin{enumerate}
  \item Store hashed credentials in a password file.
  \item User presents username and password.
  \item Check if username is present.
  \item Apply $\text{Hash}()$ to password and check if this matches the stored hash.
  \item Grant or deny access.
\end{enumerate}

The password file remains a valuable target.

\textbf{Offline dictionary attack} - A \textit{rainbow table}, a dictionary of hash and password pairs, can be built.
The corresponding password to a stolen hash can be found.

\subsubsection{Salted Hashes}
\begin{defn}
  \textbf{Salted hashing}:
  \[
    \text{Hash}(\text{plaintext } \mid \text{ salt}) = \text{ hashvalue}
  \]
  The \textbf{salt} is a cryptographically random string.
\end{defn}

\begin{enumerate}
  \item Store salted hashed credentials in a password file.
  \item User presents username and password.
  \item Check if username is present.
  \item Find user salt, see if $\text{Hash}(\text{plaintext } \mid \text{ salt})$ matches the entry for that user.
  \item Grant or deny access.
\end{enumerate}

Password file is less valuable.
A different dictionary is needed for every possible salt.
An attack against a specific user is still practical if given the salt for that user.

\subsection{Linux Password File}
\texttt{username:password-data:parameters} in \texttt{/etc/passwd}
\begin{itemize}
  \item \texttt{password-data}: \texttt{\$hash-function-id\$salt\$password}:
    \begin{itemize}
      \item \texttt{*} - disabled.
      \item \texttt{hash-function-id}:
      \begin{itemize}
        \item \texttt{1} - md5.
        \item \texttt{2a, 2y} - Blowfish.
        \item \texttt{5} - sha256.
        \item \texttt{6} - sha512.
      \end{itemize}
    \end{itemize}
  \item \texttt{parameters}, e.g.\ \texttt{:16826:0:99999:7:::}:
    \begin{itemize}
      \item \texttt{16826} - days since last change.
      \item \texttt{0} - can be changed at any time.
      \item \texttt{99999} - does not have to be changed.
      \item \texttt{7} - warn 1 week before expiry.
    \end{itemize}
\end{itemize}

\subsection{Usability}
\begin{itemize}
  \item It is hard for humans to choose and remember good passwords.
  \item Dictionary attacks can start from a vocabulary of common, memorable words, then apply \textit{password-mangling} rules to generate realistic variants (e.g.\ \textit{leetspeak}).
  \item Users tend to reuse passwords.
  \item Complex password rules are a burden.
  \item Security questions are dangerous - answers can be found via social media.
  \item Password hints should be avoided - reminders may give away password too easily.
\end{itemize}

\subsection{Online Dictionary Attack}
\begin{itemize}
  \item Attacker submits username/password combinations to a running authentication system.
  \item Usernames are relatively easy to find.
  \item Previously used passwords are easy to find - password lists from hacked websites can be found in the public domain, or purchased from the dark web.
  \item Defenses:
    \begin{itemize}
      \item Limit number of tries per username or per IP.
      \item CAPTCHAs.
      \item \textit{Honeypot} passwords - create fake, easy to guess accounts and alert if accessed.
    \end{itemize}
\end{itemize}

\subsection{Alternatives to Passwords}
\begin{itemize}
  \item Hardware tokens from banks, one-time-password booklets from governments - expensive and hard to replace.
  \item Second-factor authentication via mobile phone - SMS deprecated by NIST.
  \item Biometric authentication - can spoof.
  \item Authentication via RFID tags - risk of theft, proximity attacks.
  \item Passwordless authentication (e.g.\ temporary pin via email) - email must be secure.
\end{itemize}

\subsection{Best Practices}
\begin{itemize}
  \item Filters to ensure users select long, random-looking passwords.
  \item Do not ask users to change passwords often.
  \item Store salted hashes in a protected file.
  \item Do not fail with ``user not found''.
  \item Ask additional security questions or CAPTCHA after a few failed login attempts for the same username, or same IP.
  \item Block account or requests from the same IP after many failed attempts.
  \item Show information about last login or notify user via email upon successful login.
  \item Password managers.
  \item Login using a trusted authentication provider (e.g.\ OAuth).
\end{itemize}

\section{Penetration Testing (Pentesting)}
\begin{defn}
  \textbf{Penetration testing} is when you pay someone to break into your system/organization and report weaknesses.
\end{defn}
Pentesting allows us to both find weaknesses in our web applications, and to gain insight into an attacker's approach.

It is important to scope the pentesting exercise to avoid operational damage.

Access to information:
\begin{itemize}
  \item \textbf{Black box} - no information.
  \item \textbf{Gray box} - selected information to help focus exercise.
  \item \textbf{White box} - access to source code, system architecture, protocols, valid accounts, etc.
\end{itemize}

It is hard to ensure the pentester has tried hard enough:
\begin{itemize}
  \item Play pentesting teams against each other.
  \item Pentesting certifications.
  \item Penetration Testing Execution Standard (PTES).
\end{itemize}

\subsection{PTES}
Key steps:
\begin{enumerate}
  \item Pre-engagement interactions.
  \item Intelligence gathering.
  \item Threat modelling.
  \item Vulnerability analysis.
  \item Exploitation.
  \item Post-exploitation.
  \item Reporting.
\end{enumerate}

\subsubsection{Intelligence Gathering}
\textbf{Passive} information gathering:
\begin{itemize}
  \item Aim to build a DFD, network map, architectural diagram, organization chart, sociogram of target, etc.
  \item Avoid interaction with target to avoid suspicion.
  \item Look for publicly available information about target:
    \begin{itemize}
      \item Online presence of company and/or employees.
      \item Web presence - cache, archive, source code, protocols, uptime statistics, domain registrar, DNS.
      \item Comments in open source code used by target - open bugs, hardcoded credentials in old versions.
      \item Even publicly accessible data may be protected by law.
    \end{itemize}
\end{itemize}

\subsubsection{Google Hacking}
\begin{itemize}
  \item Advanced operators:
    \begin{itemize}
      \item \texttt{ext:pdf} - search for files with specific extensions.
      \item \texttt{site:example.com} - search within given website.
      \item \texttt{"index.html" inurl: -html} - search inside url or title or body of a page.
    \end{itemize}
  \item Identifying potential targets using operators:
    \begin{itemize}
      \item \texttt{intitle:index.of "parent directory"} - locate unintentional exposed directory listings.
      \item \texttt{allintext:"Powered by phpbb"}, \texttt{inurl:index.asp} - locate sites  running vulnerable software.
    \end{itemize}
  \item Cache search:
    \begin{itemize}
      \item Access code of webpage without accessing the actual target.
      \item Use proxy to prevent loading uncached elements.
    \end{itemize}
\end{itemize}

\subsubsection{Intelligence Gathering}
\textbf{Active} information gathering:
\begin{itemize}
  \item Contacting target services.
  \item \texttt{dig} query to discover the version of the DNS server.
  \item Verify potential email addresses and usernames.
  \item Determine network perimeter:
    \begin{itemize}
      \item Accessing target directly or via firewall?
      \item \texttt{traceroute} to reverse DNS and identify intermediate hosts.
    \end{itemize}
  \item Probe the network:
    \begin{itemize}
      \item Host discovery - identify active subnet addresses.
      \item Port scanning - identify ports accepting communication.
      \item OS fingerprinting - identify target OS.
      \item Identify services - observe identifying information from a service.
    \end{itemize}
\end{itemize}

\subsubsection{Vulnerability Analysis}
\begin{itemize}
  \item Unpatched systems:
    \begin{itemize}
      \item Search in the CVE database for vulnerabilities affecting any identified component.
      \item Scan for known vulnerabilities and exposures using tools.
    \end{itemize}
  \item Patched systems:
    \begin{itemize}
      \item Perform code review of source code.
      \item Try to trigger unknown vulnerabilities.
    \end{itemize}
  \item Obtain credentials:
    \begin{itemize}
      \item Investigate password policies, identify default passwords.
      \item Look for published password hashes.
    \end{itemize}
\end{itemize}

\subsubsection{Exploitation}
\begin{itemize}
  \item Exploit identified vulnerabilities.
  \item Use valid credentials, or brute-force.
  \item Run publicly available exploits, or build your own.
\end{itemize}

\subsubsection{Post-Exploitation}
\begin{itemize}
  \item Privilege escalation:
    \begin{itemize}
      \item \textit{Pass-the-hash} - send the admin's hashed password to an internal authentication server.
      \item Confused deputy - exploit application that runs as root.
    \end{itemize}
  \item Steal data - explore disk, keylogger, capture screenshots, etc.
  \item Send data back to hacker.
  \item Pivot - exploit other targets on LAN.
  \item Maintain access - open backdoors, create new accounts to login to later.
  \item Cover tracks - manipulate logs, install rootkits to hide backdoors.
\end{itemize}

\section{Network Security}
\subsection{The Internet}
\begin{itemize}
  \item \textbf{Host} - provider/consumer of \textit{services}, reachable via an IP address.
  \item \textbf{Autonomous Systems} (AS) - network of networks, controls a range of IPs.
  \item TCP/IP protocol stack.
  \item Packet switched:
    \begin{itemize}
      \item \textbf{Packet} (or \textbf{datagram}) - message that is sent as a single unit on the network, composed of headers and a payload.
      \item Each packet needs to be routed between endpoints.
    \end{itemize}
\end{itemize}

\subsection{IP Addresses}
\begin{itemize}
  \item Different network services are multiplexed through the same IP address using ports.
  \item One machine can have multiple IPs:
    \begin{itemize}
      \item Across networks.
      \item Across network interfaces.
    \end{itemize}
  \item Multiple machines may share the same IP.
\end{itemize}

\subsection{Network Intermediaries}
\begin{itemize}
  \item \textbf{Router}:
    \begin{itemize}
      \item Connects two different networks.
      \item Does not modify packet addresses.
    \end{itemize}
  \item \textbf{Network Address Translator} (NAT):
    \begin{itemize}
      \item Exposes a local network via the ports of 1 IP address.
      \item Modifies packet's IP addresses to affect the mapping.
    \end{itemize}
  \item \textbf{Proxy}:
    \begin{itemize}
      \item Networks A and B communicate to a proxy, not directly to each other.
      \item 2 independent packets.
    \end{itemize}
\end{itemize}

\subsection{Network Capabilities}
\begin{itemize}
  \item \textbf{Participant} - sends and receives legitimate packets that respect the protocol, e.g.\ web browser.
  \item \textbf{Eavesdropper} - reads packets sent to others, cannot (or will not) participate, e.g.\ wiretapper.
  \item \textbf{Off-path} - participates and creates arbitrary packets, e.g.\ ethernet.
  \item \textbf{Man in the Middle} (MITM) - participates, reads, modifies, creates and deletes packets, e.g.\ router.
\end{itemize}

\subsection{Layers and Protocols}
See figure \ref{fig:layers}.
\begin{figure}[htb!]
  \centering
  \caption{Layers and protocols involved in networking.}
  \label{fig:layers}
  \includegraphics[scale=0.4]{layers}
\end{figure}

\subsection{Datagram}
See figure \ref{fig:datagram}.
\begin{figure}[htb!]
  \centering
  \caption{Datagram encapsulation.}
  \label{fig:datagram}
  \includegraphics[scale=0.4]{datagram}
\end{figure}

\subsection{Processing at Different Layers}
See figure \ref{fig:processing}.
\begin{figure}[htb!]
  \centering
  \caption{Processing involved at different layers of networking.}
  \label{fig:processing}
  \includegraphics[scale=0.4]{processing}
\end{figure}

\subsection{Local Area Network}
\begin{itemize}
  \item Based on a broadcast medium, such as cable or wireless.
  \item The network interface of each host has a Media Access Control (MAC) address.
  \item Messages on the LAN are sent based on MAC addresses.
  \item Dynamic Host Configuration Protocol (DHCP) tells new hosts their IP and other configuration information.
  \item Address Resolution Protocol (ARP) used to find the MAC of an IP on the same LAN.
\end{itemize}

LAN relies on the broadcast medium, requiring a minimum packet size in order to store conflict resolution requirements.
If not initialized, this could lead to data disclosure.

\subsection{MAC Flooding}
\begin{itemize}
  \item Network switches cache port-MAC associations.
  \item The attacker floods the switch with numerous invalid source MACs, consuming the limited memory.
  \item Since there are no longer entries for valid hosts, the switch is forced to broadcast traffic to all hosts.
  \item The attacker can now sniff the packets.
  \item Countermeasures:
    \begin{itemize}
      \item Port security - limit ability to flood caches.
      \item Keep track of authorized MAC addresses in the system.
    \end{itemize}
\end{itemize}

\subsection{ARP Poisoning}
\begin{itemize}
  \item The switch needs to find the MAC corresponding to an IP.
  \item The attacker spoofs the MAC of the victim, and replies like the victim does.
  \item The message is forwarded to both ports that replied.
  \item The attacker is able to read the message.
  \item Countermeasures:
    \begin{itemize}
      \item Static ARP rules.
      \item Spoofed ARP message detection.
    \end{itemize}
\end{itemize}

\subsection{Internet Protocol}
\begin{itemize}
  \item Packets are delivered between \textit{Source} and \textit{Destination} hosts via the protocol.
  \item The structure of IP addresses guides routing.
  \item The protocol is best effort, it may drop or reorder packets.
  \item The packet may be fragmented when it transmits on networks with smaller packet sizes:
    \begin{itemize}
      \item Various OSs treat duplicate IP fragments in different ways - OS fingerprinting.
    \end{itemize}
  \item Time to live (TTL) is used to discard packets that take too many steps to reach the destination:
    \begin{itemize}
      \item The TTL is decremented at each step.
      \item At 0, the packet is discarded and an ICMP error message is sent to the source.
    \end{itemize}
\end{itemize}

\subsection{Traceroute}
\begin{itemize}
  \item The traceroute algorithm uses TTL to identify the hosts/routers on the path to the target.
  \item Packets with increasing TTL are sent until the destination is reached.
  \item Each ICMP error message should be from a host on the path to the destination.
\end{itemize}


\end{document}
