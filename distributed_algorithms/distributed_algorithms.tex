\documentclass[11pt]{article}
\usepackage{fullpage}
\usepackage{amsthm}
\usepackage{amsmath} \usepackage{amssymb}
\usepackage{graphicx}

\graphicspath{ {./imgs/} }

\setlength{\parindent}{0pt}

\title{Distributed Algorithms (CO347)}
\author{Michael Tsang}

\newtheorem{defn}{Definition}
\newtheorem{eg}{Example}
\newtheorem{theo}{Theorem}
\newtheorem{lem}{Lemma}

\begin{document}

\maketitle
\section{Motivation}
\subsection{Why a Distributed System?}
\begin{itemize}
  \item Computation and data distribution.
  \item Performance through parallelisation and distribution.
  \item Reliability and availability through replication, fault-tolerance, security.
  \item Scalability, modularity, evolvability.
  \item Heterogenity of hardware and software.
\end{itemize}

\subsection{What is a Distributed System?}
\begin{itemize}
  \item A set of processes connected by a network.
    \begin{itemize}
      \item Machines can be located across the planet.
      \item Communicate via message passing, no shared memory.
    \end{itemize}
  \item No common physical clock.
    \begin{itemize}
      \item No total order on events by time.
    \end{itemize}
\end{itemize}

\subsection{Assumptions}
When designing distributed systems, we have to consider different assumptions.

\subsubsection{Timing}
\begin{itemize}
  \item \textbf{Synchronous systems}:
    \begin{itemize}
      \item Upper bound on process delays.
      \item Upper bound on time for a message to be delivered.
      \item Can translate results to asynchronous models.
    \end{itemize}
  \item \textbf{Asynchronous systems}:
    \begin{itemize}
      \item Processes and communication take arbitrary time.
      \item No assumption that processes have physical clocks, but can be useful to use logical clocks.
    \end{itemize}
  \item \textbf{Partially synchronous systems}:
    \begin{itemize}
      \item Real-world systems are mostly synchronous with asynchronous periods.
      \item Assume the system is eventutally synchronous.
    \end{itemize}
\end{itemize}

\subsubsection{Failures}
\begin{itemize}
  \item \textbf{No failures}:
    \begin{itemize}
      \item Unrealistic.
    \end{itemize}
  \item \textbf{Process failure}:
    \begin{itemize}
      \item e.g.\ software bugs; OS/user termination; OS failure.
      \item Process stops sending messages it is supposed to send.
      \item Process sends messages it is not supposed to send.
    \end{itemize}
  \item \textbf{Link failure}:
    \begin{itemize}
      \item e.g.\ cable, router, wireless, adversary.
      \item Inter-process communication failure, e.g.\ non-delivery of messages; corrupt messages; duplicates.
      \item Need for reliable protocols and secure channels.
      \item Partitioned networks.
    \end{itemize}
\end{itemize}

\subsubsection{Failure Classes}
\begin{itemize}
  \item \textbf{Process crash failure} (crash-stop failure):
    \begin{itemize}
      \item Process halts and does not perform further action.
      \item Different types:
        \begin{itemize}
          \item \textbf{Fail-stop} - can be reliably detected by other processes.
          \item \textbf{Fail-silent} - cannot be reliably detected.
          \item \textbf{Fail-noisy} - detection takes time.
          \item \textbf{Fail-recovery} - crashed processes can recover
        \end{itemize}
      \item Non-faulty processes are a \textbf{correct process}.
    \end{itemize}
  \item \textbf{Link failure}:
    \begin{itemize}
      \item Link goes down and stays down.
      \item Network may partition or remain connected.
    \end{itemize}
  \item \textbf{Omission failure}:
    \begin{itemize}
      \item Two types:
        \begin{itemize}
          \item \textbf{Send omission} - does not send all required messages.
          \item \textbf{Receive omission} - does not receive all require messages.
        \end{itemize}
    \end{itemize}
  \item \textbf{Byzantine failure} (fail-arbitrary):
    \begin{itemize}
      \item Arbitrary (or malicious) behaviour.
    \end{itemize}
\end{itemize}

\subsubsection{Communication}
\begin{itemize}
  \item \textbf{Asynchronous message passing}:
    \begin{itemize}
      \item The process sending a message continues after sending.
      \item We can build a \textbf{synchronous message passing} mechanism, e.g.\ process waits until message is delivered to the receiving process.
      \item We can build a \textbf{shared memory abstraction}.
    \end{itemize}
  \item \textbf{Reliable message communication}:
    \begin{itemize}
      \item Assume that messages sent using reliable protocol.
      \item Communications can still fail.
      \item Use TCP to send messages, simulate network failure by dropping messages in software.
    \end{itemize}
  \item \textbf{Message delays are bounded}:
    \begin{itemize}
      \item Processes timeout if message is delayed too long.
    \end{itemize}
\end{itemize}

\end{document}
